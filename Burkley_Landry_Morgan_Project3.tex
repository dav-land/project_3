\documentclass[12pt]{article}   % use documentclass amsart too if you want
\usepackage{amsmath,amsthm,amssymb}
\usepackage[margin=1in]{geometry}
\usepackage{color}
\usepackage{hyperref}
\usepackage{graphicx}
\usepackage{fancyhdr} % COMMENT THIS OUT TO TURN OFF FANCY HEADERS

\hypersetup{
  colorlinks= true, %Colours links instead of ugly boxes
  urlcolor   = blue, %Colour for external hyperlinks
  linkcolor  = blue, %Colour of internal links
  citecolor  = blue %Colour of citations
}





\newtheorem{theorem}[equation]{Theorem}
\newtheorem{lemma}[equation]{Lemma}
\newtheorem{corollary}[equation]{Corollary}
\theoremstyle{definition}
\newtheorem{exercise}[equation]{Exercise}

\newtheorem{example}[equation]{Example}
\newtheorem{definition}[equation]{Definition}
\newtheorem{question}[equation]{Question}
\newtheorem{remark}[equation]{Remark}

\numberwithin{equation}{section}

%@@@@@@@@@@@@@@@@@@@@@@@@@@@@@@@@@@@@@@@@@@@@@@@@@@@@@@@@@@@@@@@@@@@@@@@@@@@@@@@@@@@@@@@@@

\begin{document}
\parskip10pt
\parindent0pt
\baselineskip15pt



\title{APPM 2360 PROJECT 3: THE LOTKA-VOLTERRA PREDATOR PREY MODEL}
\author{Ethan Burkley (section 251), \\ Davis Landry (section 213), Max Morgan (section 244)}


\pagestyle{fancy}
\renewcommand{\sectionmark}[1]{\markright{#1}{}}

\fancyhf{}

\rhead{\fancyplain{}{\thepage}} % predefined ()
\lhead{\fancyplain{}{\rightmark }} % 1. sectionname, 1.1 subsection name etc
%\cfoot{\fancyplain{}{\thepage}}

\maketitle

%@@@@@@@@@@@@@@@@@@@@@@@@@@@@@@@@@@@@@@@@@@@@@@@@@@@@@@@@@@@@@@@@@@@@@@@@@@@@@@@@@@@@@@@@@
\newpage
\setcounter{page}{2}
\section{Introduction} \label{APPM2360proj01sec01}

Population dynamics depend on a large number of factors, and there are many ways to model the relationship between the populations of a species of predator and prey. The populations of predators and prey are interdependent, however the size of one population causes spikes or dips in the other. Beyond just the size of the species' populations, there are many external factors that can influence the growth of these species. Including these into the model results in more accurate, yet more complex, models. Simple models exist that ignore these external factors, such as the Lotka-Volterra system. Much more complicated models take into account factors found in populations such as carrying capacity and migration.
 
%@@@@@@@@@@@@@@@@@@@@@@@@@@@@@@@@@@@@@@@@@@@@@@@@@@@@@@@@@@@@@@@@@@@@@@@@@@@@@@@@@@@@@@@@
\newpage
\setcounter{page}{3}
\section{Background} \label{APPM2360proj01sec01}



%@@@@@@@@@@@@@@@@@@@@@@@@@@@@@@@@@@@@@@@@@@@@@@@@@@@@@@@@@@@@@@@@@@@@@@@@@@@@@@@@@@@@@@@@
\newpage
\setcounter{page}{4}
\section{Problem Statement} \label{APPM2360proj01sec01}




%@@@@@@@@@@@@@@@@@@@@@@@@@@@@@@@@@@@@@@@@@@@@@@@@@@@@@@@@@@@@@@@@@@@@@@@@@@@@@@@@@@@@@@@@
\newpage
\setcounter{page}{5}
\section{Calculations and Results} \label{APPM2360proj01sec01}
The Lotka-Volterra predator prey model uses the coupled system of differential equations,
\begin{align*}
%
\frac{dx_1}{dt} & = -ax_1 + bx_1x_2 ,\ x_1(0) = x_{1,0}\\
%
\frac{dx_2}{dt} & = dx_2 - dx_1x_2 ,\ x_2(0) = x_{2,0}
%
\end{align*}

where the $x_1$ is the size of the predator population and $x_2$ is the prey population. The coefficient $a$ is the mortality rate of of the population, $b$ is the reproduction rate of the predator, $c$ is the reproduction rate of the prey, and $d$ is the mortality rate of the prey. The population of the predator is supported by interactions with prey while the prey population is impacted by interactions with the prey. 

From here, the horizontal and vertical nullclines are determined by setting the $\frac{dx_1}{dt}$ and $\frac{dx_2}{dt}$ equal to zero. Conceptually, when $\frac{dx_1}{dt}$ is equal to 0, there is no change along the horizontal axis and when $\frac{dx_2}{dt}$ is equal to 0, there is no change along the vertical axis. 
\begin{align*}
& \frac{dx_1}{dt} = \frac{dx_2}{dt} = 0
\end{align*}

Therefore the vertical nullclines are,
\begin{align*}
%
0 & =  -ax_1 + bx_1x_2 \\
%  
 & = x_1(-a + bx_2) \\
%
x_1 &= 0, \ x_2 = \frac{a}{b}  
%
\end{align*}
And the horizontal nullclines are, 
\begin{align*}
%
0 & = cx_2 - dx_1x_2 \\  
%
& =  x_2(c - dx_1) \\
%
x_2 &= 0, \ x_1 = \frac{c}{d} 
\end{align*}

<<<<<<< HEAD
The two-dimensional system of autonomous, first order differential equations can be written in matrix vector form as a vector valued function, 
\begin{align*} 
\textbf{x}'(t) 
= 
\begin{bmatrix}
     -ax_1 + bx_1x_2\\
     cx_2 - dx_1x_2\\
\end{bmatrix}
\end{align*}

The Jacobian matrix for this system is give by
\begin{align*} 
J 
=
\begin{bmatrix}
     \partial \\
     cx_2 - dx_1x_2\\
\end{bmatrix}

=======
The Equilibrium points are at $(0,0)$ and $(\frac{c}{d},\frac{a}{b})$ 
>>>>>>> 7dec0b2c93f4c62606602ed358910fbe8b84225d




%@@@@@@@@@@@@@@@@@@@@@@@@@@@@@@@@@@@@@@@@@@@@@@@@@@@@@@@@@@@@@@@@@@@@@@@@@@@@@@@@@@@@@@@@
\newpage
\setcounter{page}{6}
\section{Conclusion} \label{APPM2360proj01sec01}




%@@@@@@@@@@@@@@@@@@@@@@@@@@@@@@@@@@@@@@@@@@@@@@@@@@@@@@@@@@@@@@@@@@@@@@@@@@@@@@@@@@@@@@@@
\newpage
\setcounter{page}{7}
\section{Appendix} \label{APPM2360proj01sec01}

\end{document} 